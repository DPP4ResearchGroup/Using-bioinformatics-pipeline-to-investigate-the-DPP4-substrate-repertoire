\section{Introduction}
\lab\e{sec:intro}

Since the new millennium, Dipeptidyl peptidase – 4 (DPP4) inhibitory therapy emerges as a viable treatment option for type II diabetes and triumphed at 2006 with FDA approval of DPP4 inhibitory drug Januvia \cite{RN822}.  DPP4 as its name suggested, is a protease in nature, which is able to cleave off dipeptides from N-terminus of its substrates. Structural alternation of proteins or peptides often associates with the change of protein functions, therefore proteases is important protein function regulators, and are commonly targeted for therapeutic purposes. 
\\
In addition to the regulation of metabolism, DPP4 also regulates incredibly diverse physiological functions in human physiology. Despite many years research, DPP4 functional kingdom is not yet elucidated entirely. Revealing the entire collection of substrates of a given protease is important to understand biological relevance of protease in the context of entire organism. Nevertheless, it appears to be challenging under traditional stream of experimental biochemistry. This is largely due to strenuous experimental procedures and each reaction is only able to perform on one protease and substrate pair at the time. Consequently, early studies on proteases are commonly limited to small number of underline substrates \cite{:1992aa}. The appearance of recombinant library has expanded our understanding to exo-hydrolytic events greatly by using large recombinant terminal libraries \cite{Gupta:2010aa}. Nevertheless, recombinant library technique is useful \textit{in vitro} settings, reveals little \textit{in vivo} relevance. \textit{In vivo} hydrolytic events have more stringent restrictions, including subcellular localization of substrates and proteases, subcelluar concentrations, presence of various chaperones etc. Given the rise of modern techniques, particular the mass spectrometry has enable the validation of \textit{in vivo} hydrolysis events in a high throughput fashion. In addition, with rapidly growing data volume, "big data" \textit{in silico} analysis has started to become popular and viable, in contrary to traditional proteomics research, "big data" analysis focus on extracting features rather than depict the underline mechanisms. In this study, we will demonstrate the use of bioinformatics pipeline to mine the substrate degradome of dipeptidyl dipeptidase-4 (DPP4) in human and mice using "big data" approach and demonstrate the first \textit{in vivo}  DPP4 substrate repertoire and comprehensive global knowledgebase of DPP4.
\\
DPP4 is the archetypal member of the DPP4 protease family, and also known as CD26 cell receptor \cite{Abbott:2000qr} and is a ubiquitously expressed type II integral membrane protein \cite{Mentlein_1999}. At molecular level, DPP4 has a single 22 amino acid N-terminal hydrophobic domain that serve both signaling purpose and anchorage footing. \cite{Mentlein_1999, Abbott_2002} DPP4 exists both in membrane anchored and free soluble forms, in which the soluble form lacks of the trans-membrane domain presented in anchored form \cite{Lambeir:2001ab,Abbott:2000qr}. Catalytically, DPP4 poses a rare post proline cleavage. Proline is conserved throughout evolution in some of the most important biological molecules that carry significant functions, this category includes but not limited to growth hormons, regulatory peptides like cytokines, chemokines. This is due to proline possess a bulky aromatic like side chain, which introduce a knick in the structure and in turn preventing these peptides being hydrolysed by most common proteases. Structrully, DPP4 only possess hydrolytic ability on its dimmer forms (both homo- and hetero-dimmer) and glycosylation on DPP4 seemingly has no alteration effect on its dimmer formation and catalytic potency in anyway \cite{Aertgeerts:2004aa}. 
\\
Physiologically, DPP4 is a multifunctional protein that possess many biochemical functions including but not limited to: hydrolytic activity \cite{Lambeir:2003qf}, adenosine deaminase activity (ADA) (Kameoka,Tanaka,Nojima,Schlossman and Morimoto, 1993), interaction with extracellular matrix (Lopez-Otin and Matrisian, 2007) and regulating intracellular signalling (Lu,Hu,Wang,Qi,Gao,Li et al., 2013). DPP4 non-enzymatic functions, in most cases, involves protein-protein bindings with other proteins and regulate other proteases. Particularly, DPP4 is known to regulate matrix metalloproteinases (MMPs) and plasmin indirectly.  DPP4 also known to bind to extracellular matrix, which is believed to change the cell surface adhesion property and leads to the change in cell migratory phenotype in the context of cancers. DPP4 is also discovered as binding partners and adhesion molecule for other high profile proteins like adenosine-deaminase (ADA) (Kameoka,Tanaka,Nojima et al., 1993; De Meester,Vanham,Kestens,Vanhoof,Bosmans,Gigase et al., 1994), the Na+/H+ exchanger isoform NHE3 (Girardi,Degray,Nagy,Biemesderfer and Aronson, 2001), T cell antigen CD45 (Torimoto,Dang,Vivier,Tanaka,Schlossman and Morimoto, 1991) and many others. DPP4 is also function as a surface receptor, which is particularly identified as cell surface receptor for plasminogen (Gonzalez-Gronow,Kaczowka,Gawdi and Pizzo, 2008).  
\\
DPP4 known to regulate pleiotropic functions, in addition to metabolism, the involvement in immunity is also well known. DPP4 seems to imply in T cell proliferation due to its potent co-stimulatory property \cite{Mentlein_1999}. Early in vitro studies demonstrated the inhibition of T-cell proliferation and cytokines productions in human and led to the clinical investigations in inflammatory responses including arthritis, multiple sclerosis and inflammatory bowel diseases. The viability of applying DPP inhibitors as an intervention method against inflammatory diseases is actively explored in recent studies. 
\\
In addition to well-known DPP4 inhibitor application in mediating type II diabetes, non-selective DPP inhibitor studies also suggest the possible pharmacological application in treating inflammatory disorders. DPP4 inhibitors ease the symptoms of type II diabetes by retarding the hydrolytic potency of DPP4 and in turn preserve the concentration of glucagon-like peptide-1 (GLP-1) in circulation, which leads to the insulinotrophic effects. \cite{Ahren:2009kx} The popularity of the incretin-based therapies including DPP4 inhibitors is largely due to its disassociation with weight gain and hypoglycaemia, which often associate with other diabetic treatments and co-administratable with the conventional hypoglycaemia treatments (Roman et. 2012) DPP4 as a chronical pharmacotherapy is also attract controversy including several reports questioned the specificity of DPP4 inhibitors, which was supported by the observation of increasing incidents of pancreatic and medullary thyroid carcinoma in patients and rodent models treated with DPP4 inhibitors. (Bjerre Knudsen,Madsen,Andersen,Almholt,de Boer,Drucker et al., 2010; Vangoitsenhoven,Mathieu and Van der Schueren, 2012) Adding to the DPP4 inhibitory therapy controversy, some studies have also suggested the beneficial effects on breast and colon cancers but these findings largely acquired from in vitro or animal models, further study is required in human objects. In summary, the absence of long term follow up data in human the safety of the DPP4 inhibitor therapy cannot be guaranteed for the patient with family history of MTC or multiple endocrine neoplasia type 2. (Vangoitsenhoven,Mathieu and Van der Schueren, 2012)
With the advancement of experimental methods, there is a consistence discovery and change in belief in physiological functions DPP4 and its family members involve in. Including recent study by Tinoco,Tagore and Saghatelian (2010) using MALDI-TOF mass spectrometry reinforced the notion that DPP4 is likely to recycles peptides in kidney, which was doubted for several decades. In order to deliver the ultimate safe intervention as a therapeutic mean, a better understanding of the DPP4’s role in the complex networked human biological system is important and necessary. Depicting and understandingthe functions of the DPP4 substrates will be extremely invaluable in contributing to this goal. 
\\
Many of the modern evidence on DPP4 substrates are documented in the \textit{in vitro} and limited ~\textit{in vivo} DPP4 inhibitory studies. (Table \ref{DPP4-Sub}) To secure sufficient \textit{in vivo} evidence is important to construct a robust DPP4 knowledge network. 
\\