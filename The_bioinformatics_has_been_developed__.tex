The bioinformatics has been developed rapidly in the last decade or so. The traditional protein research involves strenuous experimental procedures and preforms on individual protein at one time. Consequently, early studies on proteases are commonly limited to small number of underline substrates \cite{:1992aa}. The appearance of recombinant library has expanded our understanding to exo-hydrolytic events greatly by using large recombinant terminal libraries \cite{Gupta:2010aa}. Nevertheless, recombinant library technique is useful \textit{in vitro} settings, reveals little \textit{in vivo} relevance. \textit{In vivo} hydrolytic events have more stringent restrictions, including subcellular localization of substrates and proteases, subcelluar concentrations, presence of various chaperones etc. Given the rise of modern techniques, particular the mass spectrometry has opened up the possibility of validation of \textit{in vivo} hydrolysis events in a high throughput fashion. In addition, with rapidly growing data volume, "big data" \textit{in silico} analysis has started to become popular, in contrary to traditional proteomics research, "big data" analysis focus on extracting features rather than depict the underline mechanisms. 