\section{Introduction}
\label{sec:intro}

Since the new millennium, Dipeptidyl peptidase (DPP) – 4 (DPP4) inhibitory therapy emerges as a viable treatment option for type II diabetes and triumphed at 2006 with FDA approval of DPP4 inhibitory drug Januvia \cite{RN822}. DPP4 as its name suggested, is a protease in nature and is able to cleave dipeptides off its substrate from N-terminus. Structural alternation of a protein or peptide often associates with the change of its functions, therefore, proteases are a major regulator of protein functions in nature and in turn they are often targeted for therapeutic purposes. 
\\
In addition to the regulation of metabolism, DPP4 also carry incredibly diverse physiological functions in human physiology. Despite many years research, DPP4 functional kingdom is not yet elucidated entirely. Revealing the entire collection of substrates of a given protease is important to understand biological relevance of DPP4 in the context of entire organism , nevertheless, this is particular difficult in the stream of experimental biochemistry. The limitations of the traditional biochemistry involves strenuous experimental procedures and each experiment only bases on individual protease and substrate pair at one time. Consequently, early studies on proteases are commonly limited to small number of underline substrates \cite{:1992aa}. The appearance of recombinant library has expanded our understanding to exo-hydrolytic events greatly by using large recombinant terminal libraries \cite{Gupta:2010aa}. Nevertheless, recombinant library technique is useful \textit{in vitro} settings, reveals little \textit{in vivo} relevance. \textit{In vivo} hydrolytic events have more stringent restrictions, including subcellular localization of substrates and proteases, subcelluar concentrations, presence of various chaperones etc. Given the rise of modern techniques, particular the mass spectrometry has offered the possibility of validation of \textit{in vivo} hydrolysis events in a high throughput fashion. In addition, with rapidly growing data volume, "big data" \textit{in silico} analysis has started to become popular and viable, in contrary to experimental proteomics methodology, "big data" analysis focus on extracting features rather than depict the underline mechanisms. In this study, we will demonstrate the use of bioinformatics pipeline to mine the substrate degradome of dipeptidyl dipeptidase-4 (DPP4) in human and mice using "big data" approach. 
\\
DPP4 is also known as CD26 cell receptor \cite{Abbott:2000qr} and is a ubiquitously expressed type II integral membrane protein \cite{Mentlein_1999}. At molecular level, DPP4 has a single 22 amino acid N-terminal hydrophobic domain that serve both signaling purpose and anchorage footing. \cite{Mentlein_1999, Abbott_2002} DPP4 exists both in membrane anchored and free soluble forms, in which the soluble form lacks of the trans-membrane domain presented in anchored form \cite{Lambeir:2001ab,Abbott:2000qr}. Catalytically, DPP4 poses a rare post proline cleavage. Proline is conserved throughout evolution in some of the most important biological molecules that carry significant functions, this category includes but not limited to growth hormons, regulatory peptides like cytokines, chemokines. This is due to proline possess a bulky aromatic like side chain, which introduce a knick in the structure and in turn preventing these peptides being hydrolysed by most common proteases. Structrully, DPP4 only possess hydrolytic ability on its dimmer forms (both homo- and hetero-dimmer) and glycosylation on DPP4 seemingly has no alteration effect on its dimmer formation and catalytic potency in anyway \cite{Aertgeerts:2004aa}. 
\\
Physiologically, both DPP4 enzymatic and non-enzymatic activites have been associated strongly in immune activation \cite{Abbott:2000qr}. DPP4 seems to imply in T cell proliferation due to its potent co-stimulatory property \cite{Mentlein_1999}. DPP4 has also been widely associated with type-2 diabetes, various chemokine biology, HIV infection, various tumor growth and many. 
\\
