\section{Methods and Materials}

\subsection{Amino acid repository}
The amino acid sequences and its corresponding post-translational modifications (PTM) in this study was collected exclusively from UniProtKB/SwissProt \cite{Magrane:2011fv}. Human and mice sequences have been acquired from nightly build, there are total of 20,138 and 16,836 entries respectively have been retrieved for this study. 
\\
\subsection{Bioinformatics workflow}
In general, each protease has a specific hydrolytic profile (consensus) that all substrates fit. In this study, we firstly established an initial DPP4 consensus from a collection of known DPP4 substrates \ref{table:}. 
\\
To explore the potential substrate degradome, previously obtained complete human and mice protein entries have been searched against following three criteria:
\begin{enumerate}
\item Protein/Peptide is no larger then 150 aa. 
\item Fit consensus profile.
\item Correct subcellualr localization.
\end{enumerate}

The subcellular localization of a protein/peptide is particular important in biological context as cellular environment is highly compartment and tightly regulated. Therefore, the presence of the protease and its \textit{in vivo} substrates have to co-exisit in the same space in order to interact. Given the DPP4 is a type II membrane protein with catalytic site reside on the extracellular space and also exisits in soluable form, current knowlege of DPP4 substrates are all secreted. 


