\section{Methods and Materials}

\subsection{Amino acid repository}
The amino acid sequences and its corresponding post-translational modifications (PTM) in this study was collected exclusively from UniProtKB/SwissProt \cite{Magrane:2011fv}. Human and mice sequences have been acquired from nightly build, there are total of 20,138 and 16,836 entries respectively have been retrieved for this study. 
\\
\subsection{Bioinformatics workflow}
In general, each protease has a specific hydrolytic profile (consensus) that all substrates fit. In this study, we firstly established an initial DPP4 consensus from a collection of known DPP4 substrates \ref{table:}. 
\\
To explore the potential substrate degradome, previously obtained complete human and mice protein entries have been searched against following three criteria:
\begin{enumerate}
\item Fit consensus profile.
\item Protein/Peptide is no larger then 150 aa. 
\item Correct subcellualr localization.
\end{enumerate}

As discussed earlier, the derived consensus has been the main selection criteria in deciding the possible DPP4 substrate, in this case, the penultimate position has been taken strong consideration given DPP4's hydrolytic profile. In addtion, the perception in DPP4 research field that DPP4 substrate is unlikely to be extremely large due to relative narrow DPP4 catalytic cavity, which restricts the large protein access. The current consensus is around 150 amino acid. Hence in this study the selection criteria has been set to 150 aa. The subcellular localization of a protein/peptide is particular important in biological context as cellular environment is highly compartment and tightly regulated. Therefore, the presence of the protease and its \textit{in vivo} substrates have to co-exist in the same space in order to interact. Given the DPP4 is a type II membrane protein with catalytic site resides on the extracellular space and also exists in soluble form, current knowledge of DPP4 substrates are all secreted. 
\\
\subsection {Diciding substrate biological pathways}
Protein sequence itself reveals little information about its biological importance. However, the biological pathways a protein underline depict precisely the biological and physiological relevance. In this study, the pathways involved by each predicted substrates has been mapped against KEGG pathway database \cite{Kanehisa01012014} according to substrate genetic ontology. In such effort, substrate UniProtKB ID has been firstly mapped to KEGG gene numbers to provide access point to KEGG pathways before the gene numbers has been subsequently linked to pathway ID. Once the KEGG pathway information has been mapped to substrates, disease information can be readily acquired via KEGG pathway ID. 
\\
\subsection{Data process and workflow automation}
UniProtKB ID has been used as an identifier to access heterogeneous data records. Two intermediate layers have been produced during this study, including UniProtKB ID to gene oncology and gene oncology to KEGG pathway layers based on text matching. Data mining, intermediate layers and pathway association have been automated using bash scripting and execution can be performed on HPC facilities. 
\\


